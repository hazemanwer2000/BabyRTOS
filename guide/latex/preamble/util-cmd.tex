%
% Command:		 Referencing.
%
\newcommand{\xrefraw}[1]{\ref{#1}}					% Referencing, number only
%
\newcommand{\xref}[1]{\Cref{#1}}					% Referencing, in general
\newcommand{\xrefrange}[2]{\Crefrange{#1}{#2}}		% Referencing, range
\newcommand{\xrefpage}[1]{\cpageref{#1}}			% Page referencing
%
%
%
%
%
%
% Command:		Footnote.
% Parameters:	{content}
%
\newcommand{\xnote}[1]{\footnote{#1}}
%
%
%
%
%
%
% Command: 		Add extra lines to column, affects single columns in multi-column pages.
% Parameters: 	{num_of_lines}
%
\newcommand{\xcolmore}[1]{\enlargethispage{#1\baselineskip}}
%
% Command: 		Break to the next page/column.
%
\newcommand{\xpgbreak}{\clearpage}
\newcommand{\xcolbreak}{\newpage}
%
%
%
%
%
%
% Command: 		Include an image.
% Parameters 	[position, b or t] {path} {size} {label} {caption}
%
% Usage:		\ximg [b] {example-image} {0.7} {example_img} {An example of a caption.}
%
\newcommand{\ximg}[5][0]{
	\ifthenelse{\equal{#1}{0}}{
		\bigskip
		\begin{center}
			\includegraphics[width=#3\linewidth]{#2}
			\captionof{figure}{#5}
			\label{#4}
		\end{center}
		\bigskip
		\par
	}{
		\begin{figure}[#1]
			\centering
			\includegraphics[width=#3\linewidth]{#2}
			\caption{#5}
			\label{#4}
		\end{figure}
	}
}
%
% Command: 		Background image.
% Parameters:	[opacity] {path} {width, 0 to 1} {anchor; e.g: north east, center} {x} {y}
%
\newcommand{\ximgoverlay}[6][1]{
	\tikz[remember picture, overlay, anchor=#4, blend mode=multiply]
	\node[opacity=#1, xshift=#5\paperwidth, yshift=#6\paperheight] at (current page.#4)
	{\includegraphics[width=#3\paperwidth]{#2}};
}
%
\newcommand{\xcover}[1]{
	\tikz[remember picture, overlay, anchor=center]
	\node[opacity=1] at (current page.center)
	{\includegraphics[width=\paperwidth]{#1}};
%	
	\thispagestyle{empty} \clearpage
}
%
%
%
%
%
% Environment: 	List items, bulleted or enumerated.
% Parameters: 	[leave empty, for larger spacing] {style} {content}
%
% Usage:		\xlist [] {\xlsdef} {
%					\xls Jim, James, Jimothy!
%					\xls Dwight Schrute!
%					\xls Pamela Anderson!
%					\xls Call, me, later!
%				}
%
\newcommand{\xlist}[3][nolistsep]{
	\ifthenelse{\equal{#1}{nolistsep}}{
		\begin{enumerate}[#1, label=#2]
	}{
		\begin{enumerate}[label=#2]
	}
	#3
	\end{enumerate}
	\par
}
%
\newcommand{\xlsnum}{\arabic*} 									% 1, 2, ..
\newcommand{\xlsrm}{\roman*} 									% i, ii, ..
\newcommand{\xlsrmcap}{\Roman*} 								% I, II, ..
\newcommand{\xlsen}{\alph*} 									% a, b, ..
\newcommand{\xlsencap}{\Alph*} 									% A, B, ..
\newcommand{\xlsencircle}{\textcircled{\scriptsize\Alph*}}
%
\newcommand{\xlsbullet}{$\bullet$}								% bullets
\newcommand{\xlsarrow}{\adfbullet{40}}							% arrows
\newcommand{\xlssquare}{\adfbullet{29}}							% squares
%
\newcommand{\xls}{\item}										% list-item
%
%
%
%
%
%
% Command:		Tabulated data.
% Parameters:	[position, b or t] {structure} {label} {caption} {content}
%
% Usage:		\xtable [b] {lL{0.6}} {table_1} {Oh, a table!} {
%					\xtbh{Day} & \xtbh{Summary} \\
%					\xtbline
%					Tuesday & Just a summary! \\
%					\xtbx{2}{c}{Multi-column!} \\
%					\xtby{2}{Multi-row!} & Just a summary! \\
%					& Just a summary!
%				}
%
\newcommand{\xtable}[5][0]{
	\ifthenelse{\equal{#1}{0}}{
		\bigskip
		\begin{center}
		\begin{tabular}{#2}
		#5
		\end{tabular}
		\captionof{table}{#4}
		\label{#3}
		\end{center}
		\bigskip
		\par
	}{
		\begin{table}[#1]
		\centering
		\begin{tabular}{#2}
		#5
		\end{tabular}
		\caption{#4}
		\label{#3}
		\end{table}
	}
}
%
% Command: 		Horizonal line in table.
%
\newcommand{\xtbline}{\hline}
%
% Command: 		Used in header row of table, consistent justification.
% Parameters: 	{content}
%
\newcommand{\xtbh}[1]{\multicolumn{1}{c}{#1}}
%
% Command: 		Used in single-cells of table to justify.
% Parameters: 	{justification} {content}
%
\newcommand{\xtbj}[2]{\multicolumn{1}{#1}{#2}}
%
% Command: 		Used to span multiple cells horizontally in a table.
% Parameters: 	{num_of_cells} {justification} {content}
%
\newcommand{\xtbx}[3]{\multicolumn{#1}{#2}{#3}}
%
% Command: 		Used to span multiple cells vertically in a table.
% Parameters: 	{num_of_cells} {content}
%
\newcommand{\xtby}[2]{\multirow{#1}{*}{#2}}
%
%
%
%
%
%
% Command:		Equations.
% Parameters:	{content}
%
\newcommand{\xequ}[1]{\begin{gather} #1 \end{gather} \par}
%
% Command:		Equation
% Parameters:	{label} {content}
%
\newcommand{\xeq}[2]{#2 \label{#1}}
%
% Command:		In-line equation.
% Parameters:	{content}
%
\newcommand{\xe}[1]{\( #1 \)}
%
%
%
%
%
%
% Command:		Quote.
% Parameters:	[author] {content}
%
\newcommand{\xquote}[2][0]{
	\ifthenelse{\equal{#1}{0}}{
		\begin{displayquote} ``\ignorespaces#2\unskip'' \end{displayquote}
	}{
		\epigraph{\texttt{`\ignorespaces#2\unskip'}}{\texttt{#1}} \bigskip
	}
	\par
}
%
% Command: 		In-line quote.
% Parameters:	{content}
%
\newcommand{\xsay}[1]{``#1''}
%
% Command: 		Narration.
% Parameters:	{content}
%
\newcommand{\xnarrate}[1]{\begin{displayquote} \ignorespaces#1\unskip \end{displayquote}}
%
% Command:		Page-filled quote.
% Parameters:	[author] {content}
%
\newcommand{\xpgquote}[2][0]{
	\clearpage
	\thispagestyle{empty}
	\null\vfill
	\ifthenelse{\equal{#1}{0}}{
		\epigraph{\large\texttt{`\ignorespaces#2\unskip'}}{}
	}{
		\epigraph{\large\texttt{`\ignorespaces#2\unskip'}}{\texttt{#1}}
	}
	\vfill
}
%
%
%
%
%
%
% Command:		Code.
% Parameters:	{language} {path}
%
\newcommand{\xcode}[2]{\medskip \lstinputlisting[language=#1]{#2} \par}
%
% Command:		Code.
% Parameters:	{language} {path} {label} {caption}
%
\ifthenelse{\equal{\xcolumns}{onecolumn}}{
	\newcommand{\xlistingpostskip}{\medskip}
}{
	\newcommand{\xlistingpostskip}{\smallskip}	
}
%
\newcommand{\xlisting}[4]{
	\bigskip
	\lstinputlisting[language=#1, label={#3}, caption={#4}, abovecaptionskip=8pt]{#2}
	\xlistingpostskip
	\par
}
%
%
%
%
%
%
% Command: 		Link.
% Parameters:	[description] {link}
%
\newcommand{\xlink}[2][0]{
	\ifthenelse{\equal{#1}{0}}{
		\url{#2}
	}{
		\href{#2}{#1}
	}
}
%